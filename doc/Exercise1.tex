\documentclass [12pt ,a4paper, naustrian]{scrartcl}

\usepackage []{inputenc}
\usepackage [naustrian]{babel}
\usepackage {lmodern}
\usepackage [T1]{fontenc}

\usepackage {fancyhdr}
\usepackage {varioref}
\usepackage[colorlinks=true,linkcolor=black]{hyperref}


\usepackage {amsmath}
\usepackage {amssymb}
\usepackage {amsthm}
\usepackage {parskip}

\usepackage{xcolor}
\usepackage[pdftex]{graphicx}

\usepackage{listings}
\usepackage{color}

\definecolor{dkgreen}{rgb}{0,0.6,0}
\definecolor{gray}{rgb}{0.5,0.5,0.5}
\definecolor{mauve}{rgb}{0.58,0,0.82}

{
\lstset{frame=tb,
  language=Java,
  aboveskip=3mm,
  belowskip=3mm,
  showstringspaces=false,
  columns=flexible,
  basicstyle={\small\ttfamily},
  numbers=none,
  numberstyle=\tiny\color{gray},
  keywordstyle=\color{blue},
  commentstyle=\color{dkgreen},
  stringstyle=\color{mauve},
  breaklines=true,
  breakatwhitespace=true,
  tabsize=2,
  showspaces=false,
  showtabs=false,
}

\DeclareMathOperator{\spn}{span}

\theoremstyle{plain}
\newtheorem{thm}{Theorem}[section]
\newtheorem{lem}[thm]{Lemma}
\newtheorem{prop}[thm]{Proposition}

\theoremstyle{definition}
\newtheorem{defn}[thm]{Definition}
\newtheorem{conj}[thm]{Vermutung}
\newtheorem{bsp}[thm]{Beispiel}
\newtheorem{bem}[thm]{Remark}

\theoremstyle{remark}
\newtheorem*{note}{Notiz}
\newtheorem{case}{Case}

\author{Christoph Hofer\\ 0955139 \and Stefan Lew \\ 0856722}

\title{Mitzi - Exercise 1}
\fancyhf{}
\pagestyle {fancy}
\lhead {\textsc {\nouppercase{\leftmark}}}
\setlength\headheight{15pt}
\lfoot {Christoph Hofer, Stefan Lew}
\rfoot {Seite \thepage}
\renewcommand{\footrulewidth}{0.5 pt}
\renewcommand{\headheight}{29 pt}

\begin{document}
\maketitle
\newpage
\tableofcontents
\newpage
\section{Implementation of chess pieces}
	The different chess pieces and the two are represented as an \verb+enum+ in \verb+Piece.java+ and \verb+Side.java+. The class \verb+Side+ additionally provides methods for getting the opposite side and the sign of the color (+ for white and - for black), which is particularly important board evaluation and negamax algorithm.
	
\section{Implementation of a simple chess board}
	The chess board is implemented via the class \verb+GameState+. The class stores the position of the pieces, counts the \verb+full_move_clock+, \verb+halfe_move_clock+ and saves the history of all played moves. A move can be performed via the method \verb+doMove()+ which additionally checks if the move is valid or not.	
	
	\subsection{Implementation of the position class}
		The class \verb+Position+ contains the main information of the chess board. The class contains among others:
		\begin{itemize}
			\item \verb+side_board+: An array of 65 \verb+Sides+, representing the color (side) if the several pieces.
			\item \verb+piece_board+: An array of 65 \verb+Pieces+, representing the piece on a the different squares.
			\item \verb+castling+: An array, which contains the square where the king can castle. It contains -1, if it is not possible.
			\item \verb+en_passant_target+: The square, where the en-passant target is positioned.
			\item \verb+active_color+: The side, which has to move.
			
		\end{itemize}
		The arrays contains \verb+null+ if at a square is no pieces. The additional entry    is reserved for illegal squares and is always set to \verb+null+. Furthermore the class stores data, which is computed once and reuses them.
		
		The class is able to:
		\begin{itemize}
			\item read and set \verb+Pieces+ on the board.
			\item reset the board to initial state. 
			\item compute an copy of the board, where only the necessary members are copied.
			\item compute all valid moves for the active side and for each square.
			\item perform a given move
			\item check if a move is valid
			\item check if a move is a hit
			\item check if castling for a side is possible
			\item check if the current position is a check, mate or stale mate position.
			\item return a string representation of the position (FEN notation, \url{see http://en.wikipedia.org/wiki/Forsyth-Edwards_Notation})
			
		\end{itemize}
		
	\subsection{Representation of squares}
	\label{sec:squares}
	We represent the squares as integer, however we do not use the usual notation 1,2,3, \ldots , but we use the so called ICCF numeric notation (see \url{http://en.wikipedia.org/wiki/ICCF_numeric_notation}). The square a1 has a value of 11, the square b3 of 23\ldots . The class \verb+SquareHelper+ provides methods to work with the notation:
	\begin{itemize}
		\item conversion: \verb+int+ $\leftrightarrow$ \verb+[row][column]+
		\item check if a square is black or white.
		\item check if a square is valid.
		\item conversation to string representation.
		\item receiving squares in a given \verb+Direction+.
	\end{itemize}
		
	\subsection{The enum Direction}
		The enum \verb+Direction+ contains the offset for the integer value of the square for each direction. Since the knight does not use the usual directions, it needs a separate offset. To simplify the code, an additional function was generated to return the capturing direction for a pawn for a certain side.

\section{Implementation of chess moves}
		The class \verb+Move+ implements a move in a chess game. A move consists of 
		\begin{itemize}
			\item the source square
			\item the destination square
			\item a \verb+Piece+ representing the promotion of the pawn.
		\end{itemize}
		
\section{The random chess player}
	The random chess player implemented in \verb+RandyBrain+ uses the function \verb+search+ to choose randomly a possible move.

\section{The Code}

\subsubsection*{ChessGame.java}
\lstinputlisting{../src/mitzi/ChessGame.java}

\subsubsection*{Piece.java}
\lstinputlisting{../src/mitzi/Piece.java}

\subsubsection*{Side.java}
\lstinputlisting{../src/mitzi/Side.java}

\subsubsection*{PieceHelper.java}
\lstinputlisting{../src/mitzi/PieceHelper.java}

\subsubsection*{SquareHelper.java}
\lstinputlisting{../src/mitzi/SquareHelper.java}

\subsubsection*{IBrain.java}
\lstinputlisting{../src/mitzi/IBrain.java}

\subsubsection*{IMove.java}
\lstinputlisting{../src/mitzi/IMove.java}

\subsubsection*{IPosition.java}
\lstinputlisting{../src/mitzi/IPosition.java}

\subsubsection*{RandyBrain.java}
\lstinputlisting{../src/mitzi/RandyBrain.java}

\subsubsection*{HumanBrain.java}
\lstinputlisting{../src/mitzi/HumanBrain.java}

\subsubsection*{Move.java}
\lstinputlisting{../src/mitzi/Move.java}

\subsubsection*{Direction.java}
\lstinputlisting{../src/mitzi/Direction.java}

\subsubsection*{Position.java}
\lstinputlisting{../src/mitzi/Position.java}



\end{document}
